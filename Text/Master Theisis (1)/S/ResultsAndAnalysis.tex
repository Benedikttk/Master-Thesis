\chapter{Analysis and Discussion}

\section{KNN}

The clustering analysis was conducted using the K-Means algorithm to distinguish between different groups in the dataset. To optimize the classification, different values of \( k \) were tested, and the optimal number of clusters was selected based on maximizing the silhouette score. The identified cluster, assumed to represent \ce{^{10}Be}, initially exhibited a high silhouette score of \( 0.7256 \), indicating strong cohesion within the cluster and clear separation from others.

To further refine the classification, outliers were removed by filtering data points based on their Euclidean distance from the cluster centroid. The effect of this filtering is evident in the reduction of both the mean distance to the centroid and the variance of these distances. Before filtering, the mean distance to the centroid was \( 13.0333 \pm 0.3656 \), with a variance of \( 114.1273 \pm 5.5262 \). After removing outliers, the mean distance decreased to \( 12.2116 \pm 0.1976 \), and the variance was significantly reduced to \( 33.0473 \pm 1.6078 \). This reduction in variance suggests that the remaining data points are more tightly clustered around the centroid, improving the precision of the classification.

Following this refinement, the filtered cluster was re-clustered using K-Means with two sub-clusters. The silhouette score after re-clustering remained high at \( 0.4249 \), confirming that the refined cluster retained its internal consistency while maintaining a clear separation from other groups. 

These results demonstrate that the applied filtering method successfully reduced the spread of the cluster while maintaining its statistical integrity, further supporting the hypothesis that this cluster represents \ce{^{10}Be}. The significant reduction in variance suggests that the remaining points are more representative of a distinct population, improving confidence in the identification of \ce{^{10}Be} within the dataset.





\newpage